Como conclusion podemos decir varias cosas. En primero lugar es muy interesante como se puede manipular el trafico de una cierta red, lo cual nos lleva a decir que la ingenieria de trafico es mas facil de llevar a cabo. Esto nos puede llegar a pensar que frente a como se encuentra hoy las redes, se podrian cambiar todos los routers por switches openflow.\\
En segundo lugar podemos decir que frente a ataques de denegacion de servicio, nos trae muchas ventajas, ya que mitigar dichos ataques se hace mucho mas facil. Pudimos ver el caso de los datagramas udp, pero tambien podemos recibir una inundacion de paquetes que consideramos validos, y frente a esa situacion, se puede tener un firewall que se encargue del problema y de esta manera se puede tener una reaccion mas efectiva y rapida frente a eso.\\
Finalmente se puede decir que respecto a las herramientas usadas, no son muy comodas, ya que para hacer las pruebas hay que levantar muchas terminales y se hace molesto. Ademas de que la maquina virtual suele trabarse y hacer lenta la ejecucion de los programas. De todas maneras se trato de llevar a cabo tests automaticos pero se hace dificil ya que no se encuentra una manera accecible de mockear la coneccion cliente-servidor y de esa manera poder generar desde un solo script la ejecucion del controlador, topologia, ventanas de wirechark o tcpdump y terminales de los host para las conecciones de iperf. \\
De todas maneras fue el tp mas interesante que hicimos en el cuatrimestre y sobre todo, desde el punto de vista de la programacion nos hace ver cosas distintas para aplicar nuestros conocimientos.
